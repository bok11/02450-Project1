
% Default to the notebook output style

    


% Inherit from the specified cell style.




    
\documentclass[11pt]{article}

    
    
    \usepackage[T1]{fontenc}
    % Nicer default font (+ math font) than Computer Modern for most use cases
    \usepackage{mathpazo}

    % Basic figure setup, for now with no caption control since it's done
    % automatically by Pandoc (which extracts ![](path) syntax from Markdown).
    \usepackage{graphicx}
    % We will generate all images so they have a width \maxwidth. This means
    % that they will get their normal width if they fit onto the page, but
    % are scaled down if they would overflow the margins.
    \makeatletter
    \def\maxwidth{\ifdim\Gin@nat@width>\linewidth\linewidth
    \else\Gin@nat@width\fi}
    \makeatother
    \let\Oldincludegraphics\includegraphics
    % Set max figure width to be 80% of text width, for now hardcoded.
    \renewcommand{\includegraphics}[1]{\Oldincludegraphics[width=.8\maxwidth]{#1}}
    % Ensure that by default, figures have no caption (until we provide a
    % proper Figure object with a Caption API and a way to capture that
    % in the conversion process - todo).
    \usepackage{caption}
    \DeclareCaptionLabelFormat{nolabel}{}
    \captionsetup{labelformat=nolabel}

    \usepackage{adjustbox} % Used to constrain images to a maximum size 
    \usepackage{xcolor} % Allow colors to be defined
    \usepackage{enumerate} % Needed for markdown enumerations to work
    \usepackage{geometry} % Used to adjust the document margins
    \usepackage{amsmath} % Equations
    \usepackage{amssymb} % Equations
    \usepackage{textcomp} % defines textquotesingle
    % Hack from http://tex.stackexchange.com/a/47451/13684:
    \AtBeginDocument{%
        \def\PYZsq{\textquotesingle}% Upright quotes in Pygmentized code
    }
    \usepackage{upquote} % Upright quotes for verbatim code
    \usepackage{eurosym} % defines \euro
    \usepackage[mathletters]{ucs} % Extended unicode (utf-8) support
    \usepackage[utf8x]{inputenc} % Allow utf-8 characters in the tex document
    \usepackage{fancyvrb} % verbatim replacement that allows latex
    \usepackage{grffile} % extends the file name processing of package graphics 
                         % to support a larger range 
    % The hyperref package gives us a pdf with properly built
    % internal navigation ('pdf bookmarks' for the table of contents,
    % internal cross-reference links, web links for URLs, etc.)
    \usepackage{hyperref}
    \usepackage{longtable} % longtable support required by pandoc >1.10
    \usepackage{booktabs}  % table support for pandoc > 1.12.2
    \usepackage[inline]{enumitem} % IRkernel/repr support (it uses the enumerate* environment)
    \usepackage[normalem]{ulem} % ulem is needed to support strikethroughs (\sout)
                                % normalem makes italics be italics, not underlines
                                
    \usepackage{rotating}
    

    
    
    % Colors for the hyperref package
    \definecolor{urlcolor}{rgb}{0,.145,.698}
    \definecolor{linkcolor}{rgb}{.71,0.21,0.01}
    \definecolor{citecolor}{rgb}{.12,.54,.11}

    % ANSI colors
    \definecolor{ansi-black}{HTML}{3E424D}
    \definecolor{ansi-black-intense}{HTML}{282C36}
    \definecolor{ansi-red}{HTML}{E75C58}
    \definecolor{ansi-red-intense}{HTML}{B22B31}
    \definecolor{ansi-green}{HTML}{00A250}
    \definecolor{ansi-green-intense}{HTML}{007427}
    \definecolor{ansi-yellow}{HTML}{DDB62B}
    \definecolor{ansi-yellow-intense}{HTML}{B27D12}
    \definecolor{ansi-blue}{HTML}{208FFB}
    \definecolor{ansi-blue-intense}{HTML}{0065CA}
    \definecolor{ansi-magenta}{HTML}{D160C4}
    \definecolor{ansi-magenta-intense}{HTML}{A03196}
    \definecolor{ansi-cyan}{HTML}{60C6C8}
    \definecolor{ansi-cyan-intense}{HTML}{258F8F}
    \definecolor{ansi-white}{HTML}{C5C1B4}
    \definecolor{ansi-white-intense}{HTML}{A1A6B2}

    % commands and environments needed by pandoc snippets
    % extracted from the output of `pandoc -s`
    \providecommand{\tightlist}{%
      \setlength{\itemsep}{0pt}\setlength{\parskip}{0pt}}
    \DefineVerbatimEnvironment{Highlighting}{Verbatim}{commandchars=\\\{\}}
    % Add ',fontsize=\small' for more characters per line
    \newenvironment{Shaded}{}{}
    \newcommand{\KeywordTok}[1]{\textcolor[rgb]{0.00,0.44,0.13}{\textbf{{#1}}}}
    \newcommand{\DataTypeTok}[1]{\textcolor[rgb]{0.56,0.13,0.00}{{#1}}}
    \newcommand{\DecValTok}[1]{\textcolor[rgb]{0.25,0.63,0.44}{{#1}}}
    \newcommand{\BaseNTok}[1]{\textcolor[rgb]{0.25,0.63,0.44}{{#1}}}
    \newcommand{\FloatTok}[1]{\textcolor[rgb]{0.25,0.63,0.44}{{#1}}}
    \newcommand{\CharTok}[1]{\textcolor[rgb]{0.25,0.44,0.63}{{#1}}}
    \newcommand{\StringTok}[1]{\textcolor[rgb]{0.25,0.44,0.63}{{#1}}}
    \newcommand{\CommentTok}[1]{\textcolor[rgb]{0.38,0.63,0.69}{\textit{{#1}}}}
    \newcommand{\OtherTok}[1]{\textcolor[rgb]{0.00,0.44,0.13}{{#1}}}
    \newcommand{\AlertTok}[1]{\textcolor[rgb]{1.00,0.00,0.00}{\textbf{{#1}}}}
    \newcommand{\FunctionTok}[1]{\textcolor[rgb]{0.02,0.16,0.49}{{#1}}}
    \newcommand{\RegionMarkerTok}[1]{{#1}}
    \newcommand{\ErrorTok}[1]{\textcolor[rgb]{1.00,0.00,0.00}{\textbf{{#1}}}}
    \newcommand{\NormalTok}[1]{{#1}}
    
    % Additional commands for more recent versions of Pandoc
    \newcommand{\ConstantTok}[1]{\textcolor[rgb]{0.53,0.00,0.00}{{#1}}}
    \newcommand{\SpecialCharTok}[1]{\textcolor[rgb]{0.25,0.44,0.63}{{#1}}}
    \newcommand{\VerbatimStringTok}[1]{\textcolor[rgb]{0.25,0.44,0.63}{{#1}}}
    \newcommand{\SpecialStringTok}[1]{\textcolor[rgb]{0.73,0.40,0.53}{{#1}}}
    \newcommand{\ImportTok}[1]{{#1}}
    \newcommand{\DocumentationTok}[1]{\textcolor[rgb]{0.73,0.13,0.13}{\textit{{#1}}}}
    \newcommand{\AnnotationTok}[1]{\textcolor[rgb]{0.38,0.63,0.69}{\textbf{\textit{{#1}}}}}
    \newcommand{\CommentVarTok}[1]{\textcolor[rgb]{0.38,0.63,0.69}{\textbf{\textit{{#1}}}}}
    \newcommand{\VariableTok}[1]{\textcolor[rgb]{0.10,0.09,0.49}{{#1}}}
    \newcommand{\ControlFlowTok}[1]{\textcolor[rgb]{0.00,0.44,0.13}{\textbf{{#1}}}}
    \newcommand{\OperatorTok}[1]{\textcolor[rgb]{0.40,0.40,0.40}{{#1}}}
    \newcommand{\BuiltInTok}[1]{{#1}}
    \newcommand{\ExtensionTok}[1]{{#1}}
    \newcommand{\PreprocessorTok}[1]{\textcolor[rgb]{0.74,0.48,0.00}{{#1}}}
    \newcommand{\AttributeTok}[1]{\textcolor[rgb]{0.49,0.56,0.16}{{#1}}}
    \newcommand{\InformationTok}[1]{\textcolor[rgb]{0.38,0.63,0.69}{\textbf{\textit{{#1}}}}}
    \newcommand{\WarningTok}[1]{\textcolor[rgb]{0.38,0.63,0.69}{\textbf{\textit{{#1}}}}}
    
    
    % Define a nice break command that doesn't care if a line doesn't already
    % exist.
    \def\br{\hspace*{\fill} \\* }
    % Math Jax compatability definitions
    \def\gt{>}
    \def\lt{<}
    % Document parameters
    \title{IMDB 5000 metadata}
    % Auther
    \author{
    Jakobe Iversen\\
    \texttt{s143262}
    \and
    Mattias Andersen\\
    \texttt{s154057}
    }
    
    
    

    % Pygments definitions
    
\makeatletter
\def\PY@reset{\let\PY@it=\relax \let\PY@bf=\relax%
    \let\PY@ul=\relax \let\PY@tc=\relax%
    \let\PY@bc=\relax \let\PY@ff=\relax}
\def\PY@tok#1{\csname PY@tok@#1\endcsname}
\def\PY@toks#1+{\ifx\relax#1\empty\else%
    \PY@tok{#1}\expandafter\PY@toks\fi}
\def\PY@do#1{\PY@bc{\PY@tc{\PY@ul{%
    \PY@it{\PY@bf{\PY@ff{#1}}}}}}}
\def\PY#1#2{\PY@reset\PY@toks#1+\relax+\PY@do{#2}}

\expandafter\def\csname PY@tok@w\endcsname{\def\PY@tc##1{\textcolor[rgb]{0.73,0.73,0.73}{##1}}}
\expandafter\def\csname PY@tok@c\endcsname{\let\PY@it=\textit\def\PY@tc##1{\textcolor[rgb]{0.25,0.50,0.50}{##1}}}
\expandafter\def\csname PY@tok@cp\endcsname{\def\PY@tc##1{\textcolor[rgb]{0.74,0.48,0.00}{##1}}}
\expandafter\def\csname PY@tok@k\endcsname{\let\PY@bf=\textbf\def\PY@tc##1{\textcolor[rgb]{0.00,0.50,0.00}{##1}}}
\expandafter\def\csname PY@tok@kp\endcsname{\def\PY@tc##1{\textcolor[rgb]{0.00,0.50,0.00}{##1}}}
\expandafter\def\csname PY@tok@kt\endcsname{\def\PY@tc##1{\textcolor[rgb]{0.69,0.00,0.25}{##1}}}
\expandafter\def\csname PY@tok@o\endcsname{\def\PY@tc##1{\textcolor[rgb]{0.40,0.40,0.40}{##1}}}
\expandafter\def\csname PY@tok@ow\endcsname{\let\PY@bf=\textbf\def\PY@tc##1{\textcolor[rgb]{0.67,0.13,1.00}{##1}}}
\expandafter\def\csname PY@tok@nb\endcsname{\def\PY@tc##1{\textcolor[rgb]{0.00,0.50,0.00}{##1}}}
\expandafter\def\csname PY@tok@nf\endcsname{\def\PY@tc##1{\textcolor[rgb]{0.00,0.00,1.00}{##1}}}
\expandafter\def\csname PY@tok@nc\endcsname{\let\PY@bf=\textbf\def\PY@tc##1{\textcolor[rgb]{0.00,0.00,1.00}{##1}}}
\expandafter\def\csname PY@tok@nn\endcsname{\let\PY@bf=\textbf\def\PY@tc##1{\textcolor[rgb]{0.00,0.00,1.00}{##1}}}
\expandafter\def\csname PY@tok@ne\endcsname{\let\PY@bf=\textbf\def\PY@tc##1{\textcolor[rgb]{0.82,0.25,0.23}{##1}}}
\expandafter\def\csname PY@tok@nv\endcsname{\def\PY@tc##1{\textcolor[rgb]{0.10,0.09,0.49}{##1}}}
\expandafter\def\csname PY@tok@no\endcsname{\def\PY@tc##1{\textcolor[rgb]{0.53,0.00,0.00}{##1}}}
\expandafter\def\csname PY@tok@nl\endcsname{\def\PY@tc##1{\textcolor[rgb]{0.63,0.63,0.00}{##1}}}
\expandafter\def\csname PY@tok@ni\endcsname{\let\PY@bf=\textbf\def\PY@tc##1{\textcolor[rgb]{0.60,0.60,0.60}{##1}}}
\expandafter\def\csname PY@tok@na\endcsname{\def\PY@tc##1{\textcolor[rgb]{0.49,0.56,0.16}{##1}}}
\expandafter\def\csname PY@tok@nt\endcsname{\let\PY@bf=\textbf\def\PY@tc##1{\textcolor[rgb]{0.00,0.50,0.00}{##1}}}
\expandafter\def\csname PY@tok@nd\endcsname{\def\PY@tc##1{\textcolor[rgb]{0.67,0.13,1.00}{##1}}}
\expandafter\def\csname PY@tok@s\endcsname{\def\PY@tc##1{\textcolor[rgb]{0.73,0.13,0.13}{##1}}}
\expandafter\def\csname PY@tok@sd\endcsname{\let\PY@it=\textit\def\PY@tc##1{\textcolor[rgb]{0.73,0.13,0.13}{##1}}}
\expandafter\def\csname PY@tok@si\endcsname{\let\PY@bf=\textbf\def\PY@tc##1{\textcolor[rgb]{0.73,0.40,0.53}{##1}}}
\expandafter\def\csname PY@tok@se\endcsname{\let\PY@bf=\textbf\def\PY@tc##1{\textcolor[rgb]{0.73,0.40,0.13}{##1}}}
\expandafter\def\csname PY@tok@sr\endcsname{\def\PY@tc##1{\textcolor[rgb]{0.73,0.40,0.53}{##1}}}
\expandafter\def\csname PY@tok@ss\endcsname{\def\PY@tc##1{\textcolor[rgb]{0.10,0.09,0.49}{##1}}}
\expandafter\def\csname PY@tok@sx\endcsname{\def\PY@tc##1{\textcolor[rgb]{0.00,0.50,0.00}{##1}}}
\expandafter\def\csname PY@tok@m\endcsname{\def\PY@tc##1{\textcolor[rgb]{0.40,0.40,0.40}{##1}}}
\expandafter\def\csname PY@tok@gh\endcsname{\let\PY@bf=\textbf\def\PY@tc##1{\textcolor[rgb]{0.00,0.00,0.50}{##1}}}
\expandafter\def\csname PY@tok@gu\endcsname{\let\PY@bf=\textbf\def\PY@tc##1{\textcolor[rgb]{0.50,0.00,0.50}{##1}}}
\expandafter\def\csname PY@tok@gd\endcsname{\def\PY@tc##1{\textcolor[rgb]{0.63,0.00,0.00}{##1}}}
\expandafter\def\csname PY@tok@gi\endcsname{\def\PY@tc##1{\textcolor[rgb]{0.00,0.63,0.00}{##1}}}
\expandafter\def\csname PY@tok@gr\endcsname{\def\PY@tc##1{\textcolor[rgb]{1.00,0.00,0.00}{##1}}}
\expandafter\def\csname PY@tok@ge\endcsname{\let\PY@it=\textit}
\expandafter\def\csname PY@tok@gs\endcsname{\let\PY@bf=\textbf}
\expandafter\def\csname PY@tok@gp\endcsname{\let\PY@bf=\textbf\def\PY@tc##1{\textcolor[rgb]{0.00,0.00,0.50}{##1}}}
\expandafter\def\csname PY@tok@go\endcsname{\def\PY@tc##1{\textcolor[rgb]{0.53,0.53,0.53}{##1}}}
\expandafter\def\csname PY@tok@gt\endcsname{\def\PY@tc##1{\textcolor[rgb]{0.00,0.27,0.87}{##1}}}
\expandafter\def\csname PY@tok@err\endcsname{\def\PY@bc##1{\setlength{\fboxsep}{0pt}\fcolorbox[rgb]{1.00,0.00,0.00}{1,1,1}{\strut ##1}}}
\expandafter\def\csname PY@tok@kc\endcsname{\let\PY@bf=\textbf\def\PY@tc##1{\textcolor[rgb]{0.00,0.50,0.00}{##1}}}
\expandafter\def\csname PY@tok@kd\endcsname{\let\PY@bf=\textbf\def\PY@tc##1{\textcolor[rgb]{0.00,0.50,0.00}{##1}}}
\expandafter\def\csname PY@tok@kn\endcsname{\let\PY@bf=\textbf\def\PY@tc##1{\textcolor[rgb]{0.00,0.50,0.00}{##1}}}
\expandafter\def\csname PY@tok@kr\endcsname{\let\PY@bf=\textbf\def\PY@tc##1{\textcolor[rgb]{0.00,0.50,0.00}{##1}}}
\expandafter\def\csname PY@tok@bp\endcsname{\def\PY@tc##1{\textcolor[rgb]{0.00,0.50,0.00}{##1}}}
\expandafter\def\csname PY@tok@fm\endcsname{\def\PY@tc##1{\textcolor[rgb]{0.00,0.00,1.00}{##1}}}
\expandafter\def\csname PY@tok@vc\endcsname{\def\PY@tc##1{\textcolor[rgb]{0.10,0.09,0.49}{##1}}}
\expandafter\def\csname PY@tok@vg\endcsname{\def\PY@tc##1{\textcolor[rgb]{0.10,0.09,0.49}{##1}}}
\expandafter\def\csname PY@tok@vi\endcsname{\def\PY@tc##1{\textcolor[rgb]{0.10,0.09,0.49}{##1}}}
\expandafter\def\csname PY@tok@vm\endcsname{\def\PY@tc##1{\textcolor[rgb]{0.10,0.09,0.49}{##1}}}
\expandafter\def\csname PY@tok@sa\endcsname{\def\PY@tc##1{\textcolor[rgb]{0.73,0.13,0.13}{##1}}}
\expandafter\def\csname PY@tok@sb\endcsname{\def\PY@tc##1{\textcolor[rgb]{0.73,0.13,0.13}{##1}}}
\expandafter\def\csname PY@tok@sc\endcsname{\def\PY@tc##1{\textcolor[rgb]{0.73,0.13,0.13}{##1}}}
\expandafter\def\csname PY@tok@dl\endcsname{\def\PY@tc##1{\textcolor[rgb]{0.73,0.13,0.13}{##1}}}
\expandafter\def\csname PY@tok@s2\endcsname{\def\PY@tc##1{\textcolor[rgb]{0.73,0.13,0.13}{##1}}}
\expandafter\def\csname PY@tok@sh\endcsname{\def\PY@tc##1{\textcolor[rgb]{0.73,0.13,0.13}{##1}}}
\expandafter\def\csname PY@tok@s1\endcsname{\def\PY@tc##1{\textcolor[rgb]{0.73,0.13,0.13}{##1}}}
\expandafter\def\csname PY@tok@mb\endcsname{\def\PY@tc##1{\textcolor[rgb]{0.40,0.40,0.40}{##1}}}
\expandafter\def\csname PY@tok@mf\endcsname{\def\PY@tc##1{\textcolor[rgb]{0.40,0.40,0.40}{##1}}}
\expandafter\def\csname PY@tok@mh\endcsname{\def\PY@tc##1{\textcolor[rgb]{0.40,0.40,0.40}{##1}}}
\expandafter\def\csname PY@tok@mi\endcsname{\def\PY@tc##1{\textcolor[rgb]{0.40,0.40,0.40}{##1}}}
\expandafter\def\csname PY@tok@il\endcsname{\def\PY@tc##1{\textcolor[rgb]{0.40,0.40,0.40}{##1}}}
\expandafter\def\csname PY@tok@mo\endcsname{\def\PY@tc##1{\textcolor[rgb]{0.40,0.40,0.40}{##1}}}
\expandafter\def\csname PY@tok@ch\endcsname{\let\PY@it=\textit\def\PY@tc##1{\textcolor[rgb]{0.25,0.50,0.50}{##1}}}
\expandafter\def\csname PY@tok@cm\endcsname{\let\PY@it=\textit\def\PY@tc##1{\textcolor[rgb]{0.25,0.50,0.50}{##1}}}
\expandafter\def\csname PY@tok@cpf\endcsname{\let\PY@it=\textit\def\PY@tc##1{\textcolor[rgb]{0.25,0.50,0.50}{##1}}}
\expandafter\def\csname PY@tok@c1\endcsname{\let\PY@it=\textit\def\PY@tc##1{\textcolor[rgb]{0.25,0.50,0.50}{##1}}}
\expandafter\def\csname PY@tok@cs\endcsname{\let\PY@it=\textit\def\PY@tc##1{\textcolor[rgb]{0.25,0.50,0.50}{##1}}}

\def\PYZbs{\char`\\}
\def\PYZus{\char`\_}
\def\PYZob{\char`\{}
\def\PYZcb{\char`\}}
\def\PYZca{\char`\^}
\def\PYZam{\char`\&}
\def\PYZlt{\char`\<}
\def\PYZgt{\char`\>}
\def\PYZsh{\char`\#}
\def\PYZpc{\char`\%}
\def\PYZdl{\char`\$}
\def\PYZhy{\char`\-}
\def\PYZsq{\char`\'}
\def\PYZdq{\char`\"}
\def\PYZti{\char`\~}
% for compatibility with earlier versions
\def\PYZat{@}
\def\PYZlb{[}
\def\PYZrb{]}
\makeatother


    % Exact colors from NB
    \definecolor{incolor}{rgb}{0.0, 0.0, 0.5}
    \definecolor{outcolor}{rgb}{0.545, 0.0, 0.0}



    
    % Prevent overflowing lines due to hard-to-break entities
    \sloppy 
    % Setup hyperref package
    \hypersetup{
      breaklinks=true,  % so long urls are correctly broken across lines
      colorlinks=true,
      urlcolor=urlcolor,
      linkcolor=linkcolor,
      citecolor=citecolor,
      }
    % Slightly bigger margins than the latex defaults
    
    \geometry{verbose,tmargin=1in,bmargin=1in,lmargin=1in,rmargin=1in}
    
    

    \begin{document}
    
    
    \maketitle
    
    

    
    \section{IMDB 5000 movie metadata}\label{imdb-5000-movie-metadata}

    We will start by loading the dataset into a pandas dataframe, and
inspect the attributes of the first entry

    \begin{Verbatim}[commandchars=\\\{\}]
{\color{incolor}In [{\color{incolor}1}]:} \PY{k+kn}{import} \PY{n+nn}{pandas} \PY{k}{as} \PY{n+nn}{pd}
        \PY{k+kn}{import} \PY{n+nn}{seaborn} \PY{k}{as} \PY{n+nn}{sns}
        \PY{k+kn}{import} \PY{n+nn}{numpy} \PY{k}{as} \PY{n+nn}{np}
        \PY{k+kn}{import} \PY{n+nn}{matplotlib}\PY{n+nn}{.}\PY{n+nn}{pyplot} \PY{k}{as} \PY{n+nn}{plt}
        \PY{k+kn}{from} \PY{n+nn}{sklearn}\PY{n+nn}{.}\PY{n+nn}{decomposition} \PY{k}{import} \PY{n}{PCA} \PY{c+c1}{\PYZsh{} Principal component analysis}
        \PY{k+kn}{from} \PY{n+nn}{sklearn}\PY{n+nn}{.}\PY{n+nn}{preprocessing} \PY{k}{import} \PY{n}{StandardScaler} \PY{c+c1}{\PYZsh{} to standardize the data}
        
        \PY{n}{df} \PY{o}{=} \PY{p}{\PYZob{}}\PY{p}{\PYZcb{}}
        \PY{n}{df}\PY{p}{[}\PY{l+s+s1}{\PYZsq{}}\PY{l+s+s1}{raw}\PY{l+s+s1}{\PYZsq{}}\PY{p}{]} \PY{o}{=} \PY{n}{pd}\PY{o}{.}\PY{n}{read\PYZus{}csv}\PY{p}{(}\PY{l+s+s2}{\PYZdq{}}\PY{l+s+s2}{./data/movie\PYZus{}metadata.csv}\PY{l+s+s2}{\PYZdq{}}\PY{p}{)}
\end{Verbatim}

    \section{1. A description of the data set.
(Jakob)}\label{a-description-of-the-data-set.-jakob}

    \subsubsection{Problem of interest}\label{problem-of-interest}

    The data consists of 28 attributes, regarded as meta data of movies.

    \subsubsection{Where the dataset was
obtained}\label{where-the-dataset-was-obtained}

    The dataset was provided by kaggle.com After obtaining the dataset, it
has been removed due to a DMCA complaint, and replaced with an
alternative dataset. This project is not compatible with the new
dataset.

    \subsubsection{What have previously been done to the
data}\label{what-have-previously-been-done-to-the-data}

    The data has been scraped from the IMDB web site, using a python script.
The data has not been pre-proccesed, which means we will expect some NaN
values.

    \subsubsection{Aim and relevant
attributes}\label{aim-and-relevant-attributes}

    Our aim is to do a classification, to predict what score a movie will
get on IMDB. We also want to do a linear regresion of the gross of a
movie. To consider which attributes are relevant, we will print out the
first movie, and look at what attributes will be relevant.

    \begin{Verbatim}[commandchars=\\\{\}]
{\color{incolor}In [{\color{incolor}2}]:} \PY{n+nb}{print}\PY{p}{(}\PY{n}{df}\PY{p}{[}\PY{l+s+s1}{\PYZsq{}}\PY{l+s+s1}{raw}\PY{l+s+s1}{\PYZsq{}}\PY{p}{]}\PY{o}{.}\PY{n}{iloc}\PY{p}{[}\PY{l+m+mi}{0}\PY{p}{]}\PY{p}{)}
\end{Verbatim}

    \begin{Verbatim}[commandchars=\\\{\}]
color                                                                    Color
director\_name                                                    James Cameron
num\_critic\_for\_reviews                                                     723
duration                                                                   178
director\_facebook\_likes                                                      0
actor\_3\_facebook\_likes                                                     855
actor\_2\_name                                                  Joel David Moore
actor\_1\_facebook\_likes                                                    1000
gross                                                              7.60506e+08
genres                                         Action|Adventure|Fantasy|Sci-Fi
actor\_1\_name                                                       CCH Pounder
movie\_title                                                            Avatar 
num\_voted\_users                                                         886204
cast\_total\_facebook\_likes                                                 4834
actor\_3\_name                                                         Wes Studi
facenumber\_in\_poster                                                         0
plot\_keywords                           avatar|future|marine|native|paraplegic
movie\_imdb\_link              http://www.imdb.com/title/tt0499549/?ref\_=fn\_t{\ldots}
num\_user\_for\_reviews                                                      3054
language                                                               English
country                                                                    USA
content\_rating                                                           PG-13
budget                                                                2.37e+08
title\_year                                                                2009
actor\_2\_facebook\_likes                                                     936
imdb\_score                                                                 7.9
aspect\_ratio                                                              1.78
movie\_facebook\_likes                                                     33000
Name: 0, dtype: object

    \end{Verbatim}

    From this data, we will chose to focus on numerical data. We do this,
partly because we feel some of the categorical data is better explained
by the numerical data, I.E. wether or not the instructor is James
cameron or not, might not be as relevant as measuring the popularity of
a director through facebook likes. This might not be true for categories
like genre, but we hope to build a more simple model of prediction, by
limiting the prediction to numerical data.

    \section{2. Detailed explanation of the
attributes(Mattias)}\label{detailed-explanation-of-the-attributesmattias}

This dataset consists of 28 different attributes and they together hold
information about a movie.

\subsection{Attribute description}\label{attribute-description}

A detailed explination of the attributes is shown in the table below.
Where each attribute can be discrete or continous, and each attributes'
objects are of different types. 

\begin{sidewaystable}
\centering
\caption{Attribute description}
\label{my-label}
\begin{tabular}{llll}
\textbf{Attribute}           & \textbf{Description}                                  & \textbf{Discrete/Continous} & \textbf{Type of attribute} \\
movie\_title                 & Holds title of the movie.                             & Discrete                    & Nominal                    \\
director\_name               & Name of director of the movie.                        & Discrete                    & Nominal                    \\
color                        & Shown in color or black and white.                    & Discrete                    & Nominal                    \\
duration                     & Duration of the movie in minutes.                     & Discrete                    & Rati                       \\
actor\_1\_name               & Name of lead actor.                                   & Discrete                    & Nominal                    \\
actor\_2\_name               & Name of second actor.                                 & Discrete                    & Nominal                    \\
actor\_3\_name               & Name of third actor.                                  & Discrete                    & Nominal                    \\
title\_year                  & Year of release.                                      & Discrete                    & Interval                   \\
genres                       & Genres the movie belongs to.                          & Discrete                    & Nominal                    \\
aspect\_ratio                & Aspect ratio                                          & Discrete                    & Nominal                    \\
facenumber\_in\_poster       & Number of faces shown in movie poster.                & Discrete                    & Ratio                      \\
language                     & Language spoken in the movie.                         & Discrete                    & Nominal                    \\
country                      & Country where the movie is filmed.                    & Discrete                    & Nominal                    \\
budget                       & Cost of the movie.                                    & Continous                   & Ratio                      \\
gross                        & Income of the movie.                                  & Continous                   & Ratio                      \\
movie\_facebook\_likes       & Count of facebook likes for the movie.                & Discrete                    & Ratio                      \\
director\_facebook\_likes    & Count of facebook likes the director has.             & Discrete                    & Ratio                      \\
actor\_1\_facebook\_likes    & Facebook likes actor 1 has.                           & Discrete                    & Ratio                      \\
actor\_2\_facebook\_likes    & Facebook likes actor 2 has.                           & Discrete                    & Ratio                      \\
actor\_3\_facebook\_likes    & Facebook likes actor 3 has.                           & Discrete                    & Ratio                      \\
cast\_total\_facebook\_likes & Total facebook likes for the whole cast of the movie. & Discrete                    & Ratio                      \\
plot\_keywords               & Keywords describing the movie.                        & Discrete                    & Nominal                    \\
content\_rating              & Rating of the movie.                                  & Discrete                    & Nominal                    \\
num\_user\_for\_reviews      & Number of users who wrote reviews.                    & Discrete                    & Ratio                      \\
num\_critic\_for\_reviews    & Number of critics who wrote reviews.                  & Discrete                    & Ratio                      \\
num\_voted\_users            & Count of users who have voted the movie.              & Discrete                    & Ratio                      \\
movie\_imdb\_link            & Holds a link to the movie on the site imdb.           & Discrete                    & Nominal                    \\
imdb\_scoreMovie             & score on IMDB                                         & Continous                   & Ordinal                   
\end{tabular}
\end{sidewaystable}

    \subsection{Summary statistics}\label{summary-statistics}

A summary over the different numerical attributes of the dataset.

    \begin{Verbatim}[commandchars=\\\{\}]
{\color{incolor}In [{\color{incolor}21}]:} \PY{n}{df}\PY{p}{[}\PY{l+s+s1}{\PYZsq{}}\PY{l+s+s1}{raw}\PY{l+s+s1}{\PYZsq{}}\PY{p}{]}\PY{o}{.}\PY{n}{describe}\PY{p}{(}\PY{p}{)}
\end{Verbatim}

            \begin{Verbatim}[commandchars=\\\{\}]
{\color{outcolor}Out[{\color{outcolor}21}]:}        num\_critic\_for\_reviews     duration  director\_facebook\_likes  \textbackslash{}
         count             4993.000000  5028.000000              4939.000000   
         mean               140.194272   107.201074               686.509212   
         std                121.601675    25.197441              2813.328607   
         min                  1.000000     7.000000                 0.000000   
         25\%                 50.000000    93.000000                 7.000000   
         50\%                110.000000   103.000000                49.000000   
         75\%                195.000000   118.000000               194.500000   
         max                813.000000   511.000000             23000.000000   
         
                actor\_3\_facebook\_likes  actor\_1\_facebook\_likes         gross  \textbackslash{}
         count             5020.000000             5036.000000  4.159000e+03   
         mean               645.009761             6560.047061  4.846841e+07   
         std               1665.041728            15020.759120  6.845299e+07   
         min                  0.000000                0.000000  1.620000e+02   
         25\%                133.000000              614.000000  5.340988e+06   
         50\%                371.500000              988.000000  2.551750e+07   
         75\%                636.000000            11000.000000  6.230944e+07   
         max              23000.000000           640000.000000  7.605058e+08   
         
                num\_voted\_users  cast\_total\_facebook\_likes  facenumber\_in\_poster  \textbackslash{}
         count     5.043000e+03                5043.000000           5030.000000   
         mean      8.366816e+04                9699.063851              1.371173   
         std       1.384853e+05               18163.799124              2.013576   
         min       5.000000e+00                   0.000000              0.000000   
         25\%       8.593500e+03                1411.000000              0.000000   
         50\%       3.435900e+04                3090.000000              1.000000   
         75\%       9.630900e+04               13756.500000              2.000000   
         max       1.689764e+06              656730.000000             43.000000   
         
                num\_user\_for\_reviews        budget   title\_year  \textbackslash{}
         count           5022.000000  4.551000e+03  4935.000000   
         mean             272.770808  3.975262e+07  2002.470517   
         std              377.982886  2.061149e+08    12.474599   
         min                1.000000  2.180000e+02  1916.000000   
         25\%               65.000000  6.000000e+06  1999.000000   
         50\%              156.000000  2.000000e+07  2005.000000   
         75\%              326.000000  4.500000e+07  2011.000000   
         max             5060.000000  1.221550e+10  2016.000000   
         
                actor\_2\_facebook\_likes   imdb\_score  aspect\_ratio  movie\_facebook\_likes  
         count             5030.000000  5043.000000   4714.000000           5043.000000  
         mean              1651.754473     6.442138      2.220403           7525.964505  
         std               4042.438863     1.125116      1.385113          19320.445110  
         min                  0.000000     1.600000      1.180000              0.000000  
         25\%                281.000000     5.800000      1.850000              0.000000  
         50\%                595.000000     6.600000      2.350000            166.000000  
         75\%                918.000000     7.200000      2.350000           3000.000000  
         max             137000.000000     9.500000     16.000000         349000.000000  
\end{Verbatim}
        
    \section{3. data visualization(Jakob) and
PCA(Mattias)}\label{data-visualizationjakob-and-pcamattias}

    Before we start, we will take alle the numeric data of the dataset, and
drop the lines with NA. We drop the lines, because it is assumed the web
scrapper made an error while scraping for the movie.

    \begin{Verbatim}[commandchars=\\\{\}]
{\color{incolor}In [{\color{incolor}23}]:} \PY{n}{df}\PY{p}{[}\PY{l+s+s1}{\PYZsq{}}\PY{l+s+s1}{numeric}\PY{l+s+s1}{\PYZsq{}}\PY{p}{]} \PY{o}{=} \PY{n}{df}\PY{p}{[}\PY{l+s+s1}{\PYZsq{}}\PY{l+s+s1}{raw}\PY{l+s+s1}{\PYZsq{}}\PY{p}{]}\PY{o}{.}\PY{n}{\PYZus{}get\PYZus{}numeric\PYZus{}data}\PY{p}{(}\PY{p}{)}
         \PY{n}{df}\PY{p}{[}\PY{l+s+s1}{\PYZsq{}}\PY{l+s+s1}{numeric}\PY{l+s+s1}{\PYZsq{}}\PY{p}{]} \PY{o}{=} \PY{n}{df}\PY{p}{[}\PY{l+s+s1}{\PYZsq{}}\PY{l+s+s1}{numeric}\PY{l+s+s1}{\PYZsq{}}\PY{p}{]}\PY{o}{.}\PY{n}{dropna}\PY{p}{(}\PY{p}{)}
         \PY{n}{df}\PY{p}{[}\PY{l+s+s1}{\PYZsq{}}\PY{l+s+s1}{numeric\PYZus{}std}\PY{l+s+s1}{\PYZsq{}}\PY{p}{]} \PY{o}{=} \PY{p}{(}\PY{n}{df}\PY{p}{[}\PY{l+s+s1}{\PYZsq{}}\PY{l+s+s1}{numeric}\PY{l+s+s1}{\PYZsq{}}\PY{p}{]} \PY{o}{\PYZhy{}} \PY{n}{df}\PY{p}{[}\PY{l+s+s1}{\PYZsq{}}\PY{l+s+s1}{numeric}\PY{l+s+s1}{\PYZsq{}}\PY{p}{]}\PY{o}{.}\PY{n}{mean}\PY{p}{(}\PY{p}{)}\PY{p}{)}\PY{o}{/}\PY{n}{df}\PY{p}{[}\PY{l+s+s1}{\PYZsq{}}\PY{l+s+s1}{numeric}\PY{l+s+s1}{\PYZsq{}}\PY{p}{]}\PY{o}{.}\PY{n}{std}\PY{p}{(}\PY{p}{)}
         \PY{n+nb}{print}\PY{p}{(}\PY{l+m+mi}{100}\PY{o}{\PYZhy{}}\PY{p}{(}\PY{n}{df}\PY{p}{[}\PY{l+s+s1}{\PYZsq{}}\PY{l+s+s1}{raw}\PY{l+s+s1}{\PYZsq{}}\PY{p}{]}\PY{o}{.}\PY{n}{shape}\PY{p}{[}\PY{l+m+mi}{0}\PY{p}{]}\PY{o}{\PYZhy{}}\PY{n}{df}\PY{p}{[}\PY{l+s+s1}{\PYZsq{}}\PY{l+s+s1}{numeric}\PY{l+s+s1}{\PYZsq{}}\PY{p}{]}\PY{o}{.}\PY{n}{shape}\PY{p}{[}\PY{l+m+mi}{0}\PY{p}{]}\PY{p}{)}\PY{o}{/}\PY{n}{df}\PY{p}{[}\PY{l+s+s1}{\PYZsq{}}\PY{l+s+s1}{raw}\PY{l+s+s1}{\PYZsq{}}\PY{p}{]}\PY{o}{.}\PY{n}{shape}\PY{p}{[}\PY{l+m+mi}{0}\PY{p}{]}\PY{p}{,}\PY{l+s+s2}{\PYZdq{}}\PY{l+s+si}{\PYZpc{} o}\PY{l+s+s2}{f the dataset remain, after dropping NA}\PY{l+s+s2}{\PYZsq{}}\PY{l+s+s2}{s.}\PY{l+s+se}{\PYZbs{}n}\PY{l+s+s2}{A list of the remaining attributes are shown below.}\PY{l+s+s2}{\PYZdq{}}\PY{p}{)}
         \PY{n+nb}{list}\PY{p}{(}\PY{n}{df}\PY{p}{[}\PY{l+s+s1}{\PYZsq{}}\PY{l+s+s1}{numeric}\PY{l+s+s1}{\PYZsq{}}\PY{p}{]}\PY{p}{)}
\end{Verbatim}

    \begin{Verbatim}[commandchars=\\\{\}]
99.75371802498513 \% of the dataset remain, after dropping NA's.
A list of the remaining attributes are shown below

    \end{Verbatim}

            \begin{Verbatim}[commandchars=\\\{\}]
{\color{outcolor}Out[{\color{outcolor}23}]:} ['num\_critic\_for\_reviews',
          'duration',
          'director\_facebook\_likes',
          'actor\_3\_facebook\_likes',
          'actor\_1\_facebook\_likes',
          'gross',
          'num\_voted\_users',
          'cast\_total\_facebook\_likes',
          'facenumber\_in\_poster',
          'num\_user\_for\_reviews',
          'budget',
          'title\_year',
          'actor\_2\_facebook\_likes',
          'imdb\_score',
          'aspect\_ratio',
          'movie\_facebook\_likes']
\end{Verbatim}
        
    The list printout of the dataset only with the numerical data, shows
that the analysation of the data will only include these 16 attributes.

    \subsection{Boxplot}\label{boxplot}

We will use boxplots, to investigete wether or not the dataset contains
outliers

    \begin{Verbatim}[commandchars=\\\{\}]
{\color{incolor}In [{\color{incolor}5}]:} \PY{o}{\PYZpc{}}\PY{k}{matplotlib} inline
        \PY{n}{plt}\PY{o}{.}\PY{n}{figure}\PY{p}{(}\PY{n}{figsize}\PY{o}{=}\PY{p}{(}\PY{l+m+mi}{30}\PY{p}{,}\PY{l+m+mi}{10}\PY{p}{)}\PY{p}{)}
        \PY{n}{sns}\PY{o}{.}\PY{n}{boxplot}\PY{p}{(}\PY{n}{data} \PY{o}{=} \PY{n}{df}\PY{p}{[}\PY{l+s+s1}{\PYZsq{}}\PY{l+s+s1}{numeric\PYZus{}std}\PY{l+s+s1}{\PYZsq{}}\PY{p}{]}\PY{o}{.}\PY{n}{iloc}\PY{p}{[}\PY{p}{:}\PY{p}{,}\PY{p}{:}\PY{l+m+mi}{8}\PY{p}{]}\PY{p}{)}\PY{p}{;}
        \PY{n}{plt}\PY{o}{.}\PY{n}{tick\PYZus{}params}\PY{p}{(}\PY{n}{labelsize}\PY{o}{=}\PY{l+m+mi}{18}\PY{p}{)}
        
        \PY{n}{plt}\PY{o}{.}\PY{n}{figure}\PY{p}{(}\PY{n}{figsize}\PY{o}{=}\PY{p}{(}\PY{l+m+mi}{30}\PY{p}{,}\PY{l+m+mi}{10}\PY{p}{)}\PY{p}{)}
        \PY{n}{sns}\PY{o}{.}\PY{n}{boxplot}\PY{p}{(}\PY{n}{data} \PY{o}{=} \PY{n}{df}\PY{p}{[}\PY{l+s+s1}{\PYZsq{}}\PY{l+s+s1}{numeric\PYZus{}std}\PY{l+s+s1}{\PYZsq{}}\PY{p}{]}\PY{o}{.}\PY{n}{iloc}\PY{p}{[}\PY{p}{:}\PY{p}{,}\PY{l+m+mi}{8}\PY{p}{:}\PY{p}{]}\PY{p}{)}\PY{p}{;}
        \PY{n}{plt}\PY{o}{.}\PY{n}{tick\PYZus{}params}\PY{p}{(}\PY{n}{labelsize}\PY{o}{=}\PY{l+m+mi}{18}\PY{p}{)}
\end{Verbatim}

    \begin{center}
    \adjustimage{max size={0.9\linewidth}{0.9\paperheight}}{output_23_0.png}
    \end{center}
    { \hspace*{\fill} \\}
    
    \begin{center}
    \adjustimage{max size={0.9\linewidth}{0.9\paperheight}}{output_23_1.png}
    \end{center}
    { \hspace*{\fill} \\}
    
    \subsection{Histogram}\label{histogram}

The boxplot tells us that all the attributes contain outliers. These
will have to be removed from the dataset, before plotting the
histograms, to give a meaningfull insight as to wether or not the
attributes are normally distributed.

    \begin{Verbatim}[commandchars=\\\{\}]
{\color{incolor}In [{\color{incolor}6}]:} \PY{k+kn}{from} \PY{n+nn}{scipy} \PY{k}{import} \PY{n}{stats}
        \PY{n}{df}\PY{p}{[}\PY{l+s+s1}{\PYZsq{}}\PY{l+s+s1}{no\PYZus{}outliers}\PY{l+s+s1}{\PYZsq{}}\PY{p}{]} \PY{o}{=} \PY{n}{df}\PY{p}{[}\PY{l+s+s1}{\PYZsq{}}\PY{l+s+s1}{numeric\PYZus{}std}\PY{l+s+s1}{\PYZsq{}}\PY{p}{]}\PY{p}{[}\PY{p}{(}\PY{n}{np}\PY{o}{.}\PY{n}{abs}\PY{p}{(}\PY{n}{stats}\PY{o}{.}\PY{n}{zscore}\PY{p}{(}\PY{n}{df}\PY{p}{[}\PY{l+s+s1}{\PYZsq{}}\PY{l+s+s1}{numeric\PYZus{}std}\PY{l+s+s1}{\PYZsq{}}\PY{p}{]}\PY{p}{)}\PY{p}{)} \PY{o}{\PYZlt{}} \PY{l+m+mi}{3}\PY{p}{)}\PY{o}{.}\PY{n}{all}\PY{p}{(}\PY{n}{axis}\PY{o}{=}\PY{l+m+mi}{1}\PY{p}{)}\PY{p}{]}
        \PY{n}{df}\PY{p}{[}\PY{l+s+s1}{\PYZsq{}}\PY{l+s+s1}{no\PYZus{}outliers}\PY{l+s+s1}{\PYZsq{}}\PY{p}{]}\PY{o}{.}\PY{n}{hist}\PY{p}{(}\PY{n}{figsize}\PY{o}{=}\PY{p}{(}\PY{l+m+mi}{15}\PY{p}{,}\PY{l+m+mi}{15}\PY{p}{)}\PY{p}{)}\PY{p}{;}
\end{Verbatim}

    \begin{center}
    \adjustimage{max size={0.9\linewidth}{0.9\paperheight}}{output_25_0.png}
    \end{center}
    { \hspace*{\fill} \\}
    
    From the histograms, it is clear that only IMDB\_score and duration is
somewhere normally distributed. IMDB\_score seemes to be a little
left-skewed, which tells us that most movies are good movies, while not
so many bad movies are included. Duration on the other hand, seems to be
slightly right hand-skewed, which tells us that only a few movies are
considered long.

\subsection{Heatmap}\label{heatmap}

We will use a heatmap, to investigate what attributes correlate with
each other.

    \begin{Verbatim}[commandchars=\\\{\}]
{\color{incolor}In [{\color{incolor}7}]:} \PY{n}{plt}\PY{o}{.}\PY{n}{figure}\PY{p}{(}\PY{n}{figsize}\PY{o}{=}\PY{p}{(}\PY{l+m+mi}{15}\PY{p}{,}\PY{l+m+mi}{10}\PY{p}{)}\PY{p}{)}
        \PY{n}{sns}\PY{o}{.}\PY{n}{heatmap}\PY{p}{(}\PY{n}{df}\PY{p}{[}\PY{l+s+s1}{\PYZsq{}}\PY{l+s+s1}{numeric}\PY{l+s+s1}{\PYZsq{}}\PY{p}{]}\PY{o}{.}\PY{n}{corr}\PY{p}{(}\PY{p}{)}\PY{p}{,}\PY{n}{annot}\PY{o}{=}\PY{k+kc}{True}\PY{p}{)}\PY{p}{;}
        \PY{n}{plt}\PY{o}{.}\PY{n}{tick\PYZus{}params}\PY{p}{(}\PY{n}{labelsize}\PY{o}{=}\PY{l+m+mi}{14}\PY{p}{)}
\end{Verbatim}

    \begin{center}
    \adjustimage{max size={0.9\linewidth}{0.9\paperheight}}{output_27_0.png}
    \end{center}
    { \hspace*{\fill} \\}
    
    From the heatmap, we can identify correlation on the dataset to be:
num\_user\_for\_reviews \& num\_voted\_users (medium)
num\_critics\_for\_reviews \& movie\_facebook\_likes (low)
actor\_1\_facebook\_likes \& cast\_total\_facebook\_likes (high) We will
now choose a minimum value for the attributes, and see if that changes
the correlation. Facebook likes (generel): 100 reviews (generel): 10
votes: 100

    \begin{Verbatim}[commandchars=\\\{\}]
{\color{incolor}In [{\color{incolor}8}]:} \PY{n}{df}\PY{p}{[}\PY{l+s+s1}{\PYZsq{}}\PY{l+s+s1}{heatmap}\PY{l+s+s1}{\PYZsq{}}\PY{p}{]} \PY{o}{=} \PY{n}{df}\PY{p}{[}\PY{l+s+s1}{\PYZsq{}}\PY{l+s+s1}{numeric}\PY{l+s+s1}{\PYZsq{}}\PY{p}{]}\PY{p}{[}\PY{p}{[}\PY{l+s+s1}{\PYZsq{}}\PY{l+s+s1}{num\PYZus{}user\PYZus{}for\PYZus{}reviews}\PY{l+s+s1}{\PYZsq{}}\PY{p}{,}\PY{l+s+s1}{\PYZsq{}}\PY{l+s+s1}{num\PYZus{}voted\PYZus{}users}\PY{l+s+s1}{\PYZsq{}}\PY{p}{,}
                                       \PY{l+s+s1}{\PYZsq{}}\PY{l+s+s1}{num\PYZus{}critic\PYZus{}for\PYZus{}reviews}\PY{l+s+s1}{\PYZsq{}}\PY{p}{,}\PY{l+s+s1}{\PYZsq{}}\PY{l+s+s1}{movie\PYZus{}facebook\PYZus{}likes}\PY{l+s+s1}{\PYZsq{}}\PY{p}{,}
                                       \PY{l+s+s1}{\PYZsq{}}\PY{l+s+s1}{actor\PYZus{}1\PYZus{}facebook\PYZus{}likes}\PY{l+s+s1}{\PYZsq{}}\PY{p}{,} \PY{l+s+s1}{\PYZsq{}}\PY{l+s+s1}{cast\PYZus{}total\PYZus{}facebook\PYZus{}likes}\PY{l+s+s1}{\PYZsq{}}\PY{p}{]}\PY{p}{]}
        \PY{n}{df}\PY{p}{[}\PY{l+s+s1}{\PYZsq{}}\PY{l+s+s1}{heatmap}\PY{l+s+s1}{\PYZsq{}}\PY{p}{]} \PY{o}{=} \PY{n}{df}\PY{p}{[}\PY{l+s+s1}{\PYZsq{}}\PY{l+s+s1}{heatmap}\PY{l+s+s1}{\PYZsq{}}\PY{p}{]}\PY{p}{[}\PY{p}{(}\PY{n}{df}\PY{p}{[}\PY{l+s+s1}{\PYZsq{}}\PY{l+s+s1}{heatmap}\PY{l+s+s1}{\PYZsq{}}\PY{p}{]}\PY{p}{[}\PY{l+s+s1}{\PYZsq{}}\PY{l+s+s1}{num\PYZus{}user\PYZus{}for\PYZus{}reviews}\PY{l+s+s1}{\PYZsq{}}\PY{p}{]} \PY{o}{\PYZgt{}} \PY{l+m+mi}{10}\PY{p}{)} \PY{o}{\PYZam{}} 
                                      \PY{p}{(}\PY{n}{df}\PY{p}{[}\PY{l+s+s1}{\PYZsq{}}\PY{l+s+s1}{heatmap}\PY{l+s+s1}{\PYZsq{}}\PY{p}{]}\PY{p}{[}\PY{l+s+s1}{\PYZsq{}}\PY{l+s+s1}{num\PYZus{}critic\PYZus{}for\PYZus{}reviews}\PY{l+s+s1}{\PYZsq{}}\PY{p}{]} \PY{o}{\PYZgt{}} \PY{l+m+mi}{10}\PY{p}{)} \PY{o}{\PYZam{}}
                                      \PY{p}{(}\PY{n}{df}\PY{p}{[}\PY{l+s+s1}{\PYZsq{}}\PY{l+s+s1}{heatmap}\PY{l+s+s1}{\PYZsq{}}\PY{p}{]}\PY{p}{[}\PY{l+s+s1}{\PYZsq{}}\PY{l+s+s1}{num\PYZus{}voted\PYZus{}users}\PY{l+s+s1}{\PYZsq{}}\PY{p}{]} \PY{o}{\PYZgt{}} \PY{l+m+mi}{100}\PY{p}{)} \PY{o}{\PYZam{}}
                                      \PY{p}{(}\PY{n}{df}\PY{p}{[}\PY{l+s+s1}{\PYZsq{}}\PY{l+s+s1}{heatmap}\PY{l+s+s1}{\PYZsq{}}\PY{p}{]}\PY{p}{[}\PY{l+s+s1}{\PYZsq{}}\PY{l+s+s1}{movie\PYZus{}facebook\PYZus{}likes}\PY{l+s+s1}{\PYZsq{}}\PY{p}{]} \PY{o}{\PYZgt{}} \PY{l+m+mi}{100}\PY{p}{)} \PY{o}{\PYZam{}}
                                      \PY{p}{(}\PY{n}{df}\PY{p}{[}\PY{l+s+s1}{\PYZsq{}}\PY{l+s+s1}{heatmap}\PY{l+s+s1}{\PYZsq{}}\PY{p}{]}\PY{p}{[}\PY{l+s+s1}{\PYZsq{}}\PY{l+s+s1}{actor\PYZus{}1\PYZus{}facebook\PYZus{}likes}\PY{l+s+s1}{\PYZsq{}}\PY{p}{]} \PY{o}{\PYZgt{}} \PY{l+m+mi}{100}\PY{p}{)} \PY{o}{\PYZam{}}
                                      \PY{p}{(}\PY{n}{df}\PY{p}{[}\PY{l+s+s1}{\PYZsq{}}\PY{l+s+s1}{heatmap}\PY{l+s+s1}{\PYZsq{}}\PY{p}{]}\PY{p}{[}\PY{l+s+s1}{\PYZsq{}}\PY{l+s+s1}{cast\PYZus{}total\PYZus{}facebook\PYZus{}likes}\PY{l+s+s1}{\PYZsq{}}\PY{p}{]} \PY{o}{\PYZgt{}} \PY{l+m+mi}{100}\PY{p}{)}\PY{p}{]}
        \PY{n+nb}{print}\PY{p}{(}\PY{l+s+s2}{\PYZdq{}}\PY{l+s+s2}{This removes}\PY{l+s+s2}{\PYZdq{}}\PY{p}{,}\PY{p}{(}\PY{p}{(}\PY{n}{df}\PY{p}{[}\PY{l+s+s1}{\PYZsq{}}\PY{l+s+s1}{numeric}\PY{l+s+s1}{\PYZsq{}}\PY{p}{]}\PY{o}{.}\PY{n}{shape}\PY{p}{[}\PY{l+m+mi}{0}\PY{p}{]}\PY{o}{\PYZhy{}}\PY{n}{df}\PY{p}{[}\PY{l+s+s1}{\PYZsq{}}\PY{l+s+s1}{heatmap}\PY{l+s+s1}{\PYZsq{}}\PY{p}{]}\PY{o}{.}\PY{n}{shape}\PY{p}{[}\PY{l+m+mi}{0}\PY{p}{]}\PY{p}{)}\PY{o}{*}\PY{l+m+mi}{100}\PY{p}{)}\PY{o}{/}\PY{n}{df}\PY{p}{[}\PY{l+s+s1}{\PYZsq{}}\PY{l+s+s1}{numeric}\PY{l+s+s1}{\PYZsq{}}\PY{p}{]}\PY{o}{.}\PY{n}{shape}\PY{p}{[}\PY{l+m+mi}{0}\PY{p}{]}\PY{p}{,}\PY{l+s+s2}{\PYZdq{}}\PY{l+s+si}{\PYZpc{} o}\PY{l+s+s2}{f the data, leaving}\PY{l+s+s2}{\PYZdq{}}\PY{p}{,}\PY{n}{df}\PY{p}{[}\PY{l+s+s1}{\PYZsq{}}\PY{l+s+s1}{heatmap}\PY{l+s+s1}{\PYZsq{}}\PY{p}{]}\PY{o}{.}\PY{n}{shape}\PY{p}{[}\PY{l+m+mi}{0}\PY{p}{]}\PY{p}{,}\PY{l+s+s2}{\PYZdq{}}\PY{l+s+s2}{rows.}\PY{l+s+s2}{\PYZdq{}}\PY{p}{)}
\end{Verbatim}

    \begin{Verbatim}[commandchars=\\\{\}]
This removes 49.72375690607735 \% of the data, leaving 1911 rows.

    \end{Verbatim}

    By setting these conditions, we removed nearly half of the dataset. This
is far from perfect, and could be done more efficently. We choose to try
and map the correlation, to get an undertanding of wether or not this
approach changes anything.

    \begin{Verbatim}[commandchars=\\\{\}]
{\color{incolor}In [{\color{incolor}9}]:} \PY{n}{plt}\PY{o}{.}\PY{n}{figure}\PY{p}{(}\PY{n}{figsize}\PY{o}{=}\PY{p}{(}\PY{l+m+mi}{10}\PY{p}{,}\PY{l+m+mi}{7}\PY{p}{)}\PY{p}{)}
        \PY{n}{sns}\PY{o}{.}\PY{n}{heatmap}\PY{p}{(}\PY{n}{df}\PY{p}{[}\PY{l+s+s1}{\PYZsq{}}\PY{l+s+s1}{heatmap}\PY{l+s+s1}{\PYZsq{}}\PY{p}{]}\PY{o}{.}\PY{n}{corr}\PY{p}{(}\PY{p}{)}\PY{p}{,}\PY{n}{annot}\PY{o}{=}\PY{k+kc}{True}\PY{p}{)}\PY{p}{;}
        \PY{n}{plt}\PY{o}{.}\PY{n}{tick\PYZus{}params}\PY{p}{(}\PY{n}{labelsize}\PY{o}{=}\PY{l+m+mi}{10}\PY{p}{)}
\end{Verbatim}

    \begin{center}
    \adjustimage{max size={0.9\linewidth}{0.9\paperheight}}{output_31_0.png}
    \end{center}
    { \hspace*{\fill} \\}
    
    From the new heatmap, it is clear that the correlation is nearly the
same. As expected the correlation has generally increased, except for
actor 1 vs cast facebook likes, which has decreased. The increased
correlation dosent seem to be of a significant magnitude.

    \subsection{Eigendecomposition}\label{eigendecomposition}

The data is preprocessed, where the eigenvectors and eigenvalues are
found with an eigendecomposition of the covariance matrix.

    \subsubsection{Covariance between
features}\label{covariance-between-features}

To be able to perform an eigendecomposition to find the eigenvalues and
eigenvectors, we first need to find the covariance between the features.

    \begin{Verbatim}[commandchars=\\\{\}]
{\color{incolor}In [{\color{incolor}16}]:} \PY{n}{mean\PYZus{}vector} \PY{o}{=} \PY{n}{np}\PY{o}{.}\PY{n}{mean}\PY{p}{(}\PY{n}{df}\PY{p}{[}\PY{l+s+s1}{\PYZsq{}}\PY{l+s+s1}{numeric\PYZus{}std}\PY{l+s+s1}{\PYZsq{}}\PY{p}{]}\PY{p}{,} \PY{n}{axis}\PY{o}{=}\PY{l+m+mi}{0}\PY{p}{)}
         \PY{n}{cov\PYZus{}matrix} \PY{o}{=} \PY{n}{np}\PY{o}{.}\PY{n}{cov}\PY{p}{(}\PY{n}{df}\PY{p}{[}\PY{l+s+s1}{\PYZsq{}}\PY{l+s+s1}{numeric\PYZus{}std}\PY{l+s+s1}{\PYZsq{}}\PY{p}{]}\PY{o}{.}\PY{n}{T}\PY{p}{)}
         \PY{n+nb}{print}\PY{p}{(}\PY{l+s+s1}{\PYZsq{}}\PY{l+s+s1}{Covariance matrix: }\PY{l+s+se}{\PYZbs{}n}\PY{l+s+s1}{\PYZsq{}}\PY{p}{,} \PY{n}{pd}\PY{o}{.}\PY{n}{DataFrame}\PY{p}{(}\PY{n}{cov\PYZus{}matrix}\PY{p}{)}\PY{p}{)}
\end{Verbatim}

    \begin{Verbatim}[commandchars=\\\{\}]
Covariance matrix: 
           0         1         2         3         4         5         6   \textbackslash{}
0   1.000000  0.227705  0.176916  0.255086  0.170198  0.468535  0.594990   
1   0.227705  1.000000  0.179734  0.125771  0.084720  0.244743  0.338038   
2   0.176916  0.179734  1.000000  0.118240  0.090733  0.139938  0.300619   
3   0.255086  0.125771  0.118240  1.000000  0.253720  0.301584  0.269455   
4   0.170198  0.084720  0.090733  0.253720  1.000000  0.147045  0.182265   
5   0.468535  0.244743  0.139938  0.301584  0.147045  1.000000  0.626948   
6   0.594990  0.338038  0.300619  0.269455  0.182265  0.626948  1.000000   
7   0.241005  0.121171  0.119741  0.490686  0.944925  0.238687  0.251940   
8  -0.034009  0.029100 -0.047619  0.105018  0.057580 -0.032254 -0.032026   
9   0.566795  0.350391  0.218311  0.207321  0.125221  0.547107  0.779925   
10  0.105681  0.068161  0.018559  0.040478  0.017086  0.100389  0.066824   
11  0.410380 -0.129422 -0.044606  0.115535  0.093742  0.052368  0.021938   
12  0.255837  0.129452  0.116900  0.554182  0.392676  0.254659  0.246660   
13  0.343881  0.366124  0.190838  0.064974  0.093131  0.212124  0.477917   
14  0.180641  0.153114  0.037871  0.047123  0.057604  0.065260  0.085485   
15  0.703969  0.214936  0.162737  0.272513  0.131778  0.368494  0.518691   

          7         8         9         10        11        12        13  \textbackslash{}
0   0.241005 -0.034009  0.566795  0.105681  0.410380  0.255837  0.343881   
1   0.121171  0.029100  0.350391  0.068161 -0.129422  0.129452  0.366124   
2   0.119741 -0.047619  0.218311  0.018559 -0.044606  0.116900  0.190838   
3   0.490686  0.105018  0.207321  0.040478  0.115535  0.554182  0.064974   
4   0.944925  0.057580  0.125221  0.017086  0.093742  0.392676  0.093131   
5   0.238687 -0.032254  0.547107  0.100389  0.052368  0.254659  0.212124   
6   0.251940 -0.032026  0.779925  0.066824  0.021938  0.246660  0.477917   
7   1.000000  0.080985  0.182288  0.029423  0.124015  0.644016  0.106259   
8   0.080985  1.000000 -0.079404 -0.021757  0.067952  0.074138 -0.064292   
9   0.182288 -0.079404  1.000000  0.071254  0.017594  0.189582  0.322522   
10  0.029423 -0.021757  0.071254  1.000000  0.046293  0.036211  0.029041   
11  0.124015  0.067952  0.017594  0.046293  1.000000  0.119739 -0.129265   
12  0.644016  0.074138  0.189582  0.036211  0.119739  1.000000  0.102060   
13  0.106259 -0.064292  0.322522  0.029041 -0.129265  0.102060  1.000000   
14  0.069675  0.016620  0.098557  0.025796  0.219779  0.064215  0.028454   
15  0.207061  0.014332  0.371970  0.053035  0.302835  0.233632  0.279478   

          14        15  
0   0.180641  0.703969  
1   0.153114  0.214936  
2   0.037871  0.162737  
3   0.047123  0.272513  
4   0.057604  0.131778  
5   0.065260  0.368494  
6   0.085485  0.518691  
7   0.069675  0.207061  
8   0.016620  0.014332  
9   0.098557  0.371970  
10  0.025796  0.053035  
11  0.219779  0.302835  
12  0.064215  0.233632  
13  0.028454  0.279478  
14  1.000000  0.110318  
15  0.110318  1.000000  

    \end{Verbatim}

    \subsubsection{Eigenvalues and
eigenvectors}\label{eigenvalues-and-eigenvectors}

Now the eigendecomposition of the covariance matrix can be performed.

    \begin{Verbatim}[commandchars=\\\{\}]
{\color{incolor}In [{\color{incolor}17}]:} \PY{n}{eigen\PYZus{}val}\PY{p}{,} \PY{n}{eigen\PYZus{}vec} \PY{o}{=} \PY{n}{np}\PY{o}{.}\PY{n}{linalg}\PY{o}{.}\PY{n}{eig}\PY{p}{(}\PY{n}{cov\PYZus{}matrix}\PY{p}{)}
         \PY{n+nb}{print}\PY{p}{(}\PY{l+s+s1}{\PYZsq{}}\PY{l+s+s1}{Eigenvalues: }\PY{l+s+se}{\PYZbs{}n}\PY{l+s+s1}{\PYZsq{}}\PY{p}{,} \PY{n}{pd}\PY{o}{.}\PY{n}{DataFrame}\PY{p}{(}\PY{n}{eigen\PYZus{}val}\PY{p}{)}\PY{p}{)}
         \PY{n+nb}{print}\PY{p}{(}\PY{l+s+s1}{\PYZsq{}}\PY{l+s+s1}{Eigenvectors: }\PY{l+s+se}{\PYZbs{}n}\PY{l+s+s1}{\PYZsq{}}\PY{p}{,} \PY{n}{pd}\PY{o}{.}\PY{n}{DataFrame}\PY{p}{(}\PY{n}{eigen\PYZus{}vec}\PY{p}{)}\PY{p}{)}
\end{Verbatim}

    \begin{Verbatim}[commandchars=\\\{\}]
Eigenvalues: 
            0
0   4.445805
1   2.136388
2   1.496763
3   0.001771
4   0.149833
5   0.245490
6   0.415714
7   0.443200
8   0.483090
9   0.591156
10  0.781094
11  1.051047
12  1.013853
13  0.995797
14  0.865921
15  0.883081
Eigenvectors: 
           0         1         2         3         4         5         6   \textbackslash{}
0  -0.362354 -0.155232 -0.327054 -0.001620 -0.331574 -0.750503 -0.041216   
1  -0.206547 -0.177457  0.272901  0.001781 -0.082209  0.022426  0.001478   
2  -0.158009 -0.094838  0.208406 -0.000006  0.057453 -0.049707 -0.031895   
3  -0.252237  0.277675  0.006544  0.111629 -0.007320 -0.052765  0.640803   
4  -0.222852  0.463064  0.164287  0.616236  0.013754 -0.008826  0.140744   
5  -0.317778 -0.135717  0.017333  0.012664  0.150519 -0.075840 -0.205560   
6  -0.385210 -0.240217  0.093722 -0.001502 -0.659253  0.483335  0.060286   
7  -0.284547  0.505361  0.148158 -0.757892  0.013455 -0.004442  0.039677   
8  -0.004622  0.161083 -0.088785  0.000465  0.027568 -0.051492 -0.042603   
9  -0.340355 -0.257060  0.074307 -0.000977  0.542281 -0.049412  0.136651   
10 -0.057873 -0.037668 -0.067696 -0.000343  0.006653  0.035775  0.009101   
11 -0.102572  0.114371 -0.667981  0.000974  0.065662  0.271071  0.102190   
12 -0.263037  0.365829  0.062501  0.182208  0.001756  0.042682 -0.686442   
13 -0.222104 -0.238787  0.272306 -0.004567  0.197252 -0.001867  0.072548   
14 -0.092043 -0.007484 -0.290456 -0.000429  0.010861  0.009207 -0.009985   
15 -0.317372 -0.122225 -0.300752  0.001010  0.289985  0.331305 -0.108632   

          7         8         9         10        11        12        13  \textbackslash{}
0  -0.116389  0.036545 -0.029760  0.071827 -0.061316  0.037905  0.081510   
1   0.068313 -0.101619 -0.703490  0.249215  0.480791 -0.061195 -0.157182   
2   0.093815 -0.012812 -0.034403 -0.273099  0.126949  0.120085  0.160521   
3   0.045459  0.056373  0.076949  0.343233 -0.138729 -0.284859 -0.135369   
4   0.040965  0.103225 -0.075917 -0.261221  0.078790  0.287897  0.113896   
5   0.715876  0.092710 -0.024031 -0.296164 -0.248506 -0.109607 -0.126009   
6  -0.143677 -0.060834  0.163757 -0.202532 -0.102664 -0.064023  0.037687   
7   0.004679  0.040474 -0.030542 -0.093371  0.023544  0.164534  0.054089   
8  -0.024626 -0.010567  0.133912 -0.320901  0.345420 -0.759305 -0.207488   
9  -0.511961 -0.285736 -0.027260 -0.303659 -0.110103 -0.017139 -0.005135   
10 -0.050465  0.029301  0.084763 -0.009318 -0.066195  0.321987 -0.900841   
11  0.222318 -0.561258 -0.183168  0.011778  0.024169  0.089325  0.090690   
12 -0.196303 -0.228572  0.107281  0.309857 -0.067118 -0.102478 -0.065726   
13  0.268416 -0.242306  0.517461  0.416090  0.199927  0.049891  0.102415   
14 -0.022946  0.190690  0.330770 -0.110311  0.682436  0.255166 -0.006485   
15 -0.101522  0.644562 -0.122173  0.240690 -0.063741 -0.081187  0.117026   

          14        15  
0  -0.164751 -0.053200  
1   0.065932 -0.090268  
2  -0.252499  0.843224  
3   0.313751  0.295252  
4  -0.229883 -0.260482  
5   0.315229 -0.105331  
6   0.030401 -0.064023  
7  -0.086317 -0.127295  
8  -0.310962 -0.060707  
9   0.185893 -0.116929  
10 -0.230444  0.076745  
11 -0.154129  0.057410  
12  0.198561  0.172609  
13 -0.342140 -0.183407  
14  0.464810  0.037680  
15 -0.255101  0.013326  

    \end{Verbatim}

    \subsection{Selecting principal
components}\label{selecting-principal-components}

The next step towards the goal of the principal component analysis is to
find the number of principal components to include in the model.

First the eigenvalues and eigenvectors are gathered in tuples.

    \begin{Verbatim}[commandchars=\\\{\}]
{\color{incolor}In [{\color{incolor}12}]:} \PY{n}{eigen\PYZus{}pairs} \PY{o}{=} \PY{p}{[}\PY{p}{(}\PY{n}{np}\PY{o}{.}\PY{n}{abs}\PY{p}{(}\PY{n}{eigen\PYZus{}val}\PY{p}{[}\PY{n}{i}\PY{p}{]}\PY{p}{)}\PY{p}{,} \PY{n}{eigen\PYZus{}vec}\PY{p}{[}\PY{p}{:}\PY{p}{,}\PY{n}{i}\PY{p}{]}\PY{p}{)} \PY{k}{for} \PY{n}{i} \PY{o+ow}{in} \PY{n+nb}{range}\PY{p}{(}\PY{n+nb}{len}\PY{p}{(}\PY{n}{eigen\PYZus{}val}\PY{p}{)}\PY{p}{)}\PY{p}{]}
         
         \PY{n}{eigen\PYZus{}pairs}\PY{o}{.}\PY{n}{sort}\PY{p}{(}\PY{p}{)}
         \PY{n}{eigen\PYZus{}pairs}\PY{o}{.}\PY{n}{reverse}\PY{p}{(}\PY{p}{)}
         
         \PY{n+nb}{print}\PY{p}{(}\PY{l+s+s1}{\PYZsq{}}\PY{l+s+s1}{Eigenvalues from highest to lowest:}\PY{l+s+s1}{\PYZsq{}}\PY{p}{)}
         \PY{k}{for} \PY{n}{eig\PYZus{}val} \PY{o+ow}{in} \PY{n}{eigen\PYZus{}pairs}\PY{p}{:}
             \PY{n+nb}{print}\PY{p}{(}\PY{n}{eig\PYZus{}val}\PY{p}{[}\PY{l+m+mi}{0}\PY{p}{]}\PY{p}{)}
\end{Verbatim}

    \begin{Verbatim}[commandchars=\\\{\}]
Eigenvalues from highest to lowest:
4.44580469646
2.1363882565
1.49676251129
1.05104701557
1.01385279214
0.995797014572
0.883080919082
0.86592055526
0.781093607224
0.591155511436
0.483090304644
0.443199530174
0.415713571524
0.245489846152
0.149832591928
0.00177127604066

    \end{Verbatim}

    \subsubsection{Explained variance}\label{explained-variance}

A plot is generated showing the explained variance.

    \begin{Verbatim}[commandchars=\\\{\}]
{\color{incolor}In [{\color{incolor}13}]:} \PY{n}{var\PYZus{}explained} \PY{o}{=} \PY{p}{[}\PY{p}{(}\PY{n}{i} \PY{o}{/} \PY{n+nb}{sum}\PY{p}{(}\PY{n}{eigen\PYZus{}val}\PY{p}{)}\PY{p}{)}\PY{o}{*}\PY{l+m+mi}{100} \PY{k}{for} \PY{n}{i} \PY{o+ow}{in} \PY{n+nb}{sorted}\PY{p}{(}\PY{n}{eigen\PYZus{}val}\PY{p}{,} \PY{n}{reverse}\PY{o}{=}\PY{k+kc}{True}\PY{p}{)}\PY{p}{]}
         \PY{n}{cum\PYZus{}var\PYZus{}explained} \PY{o}{=} \PY{n}{np}\PY{o}{.}\PY{n}{cumsum}\PY{p}{(}\PY{n}{var\PYZus{}explained}\PY{p}{)}
         
         \PY{n}{plt}\PY{o}{.}\PY{n}{figure}\PY{p}{(}\PY{n}{figsize}\PY{o}{=}\PY{p}{(}\PY{l+m+mi}{15}\PY{p}{,}\PY{l+m+mi}{5}\PY{p}{)}\PY{p}{)}
         \PY{n}{plt}\PY{o}{.}\PY{n}{bar}\PY{p}{(}\PY{n+nb}{range}\PY{p}{(}\PY{l+m+mi}{1}\PY{p}{,}\PY{l+m+mi}{17}\PY{p}{)}\PY{p}{,} \PY{n}{var\PYZus{}explained}\PY{p}{,} \PY{n}{alpha}\PY{o}{=}\PY{l+m+mf}{0.5}\PY{p}{,} \PY{n}{align}\PY{o}{=}\PY{l+s+s1}{\PYZsq{}}\PY{l+s+s1}{center}\PY{l+s+s1}{\PYZsq{}}\PY{p}{,} \PY{n}{label}\PY{o}{=}\PY{l+s+s1}{\PYZsq{}}\PY{l+s+s1}{Individual explained variance}\PY{l+s+s1}{\PYZsq{}}\PY{p}{,} \PY{n}{color}\PY{o}{=}\PY{l+s+s1}{\PYZsq{}}\PY{l+s+s1}{b}\PY{l+s+s1}{\PYZsq{}}\PY{p}{)}
         \PY{n}{plt}\PY{o}{.}\PY{n}{step}\PY{p}{(}\PY{n+nb}{range}\PY{p}{(}\PY{l+m+mi}{1}\PY{p}{,}\PY{l+m+mi}{17}\PY{p}{)}\PY{p}{,} \PY{n}{cum\PYZus{}var\PYZus{}explained}\PY{p}{,} \PY{n}{where}\PY{o}{=}\PY{l+s+s1}{\PYZsq{}}\PY{l+s+s1}{mid}\PY{l+s+s1}{\PYZsq{}}\PY{p}{,} \PY{n}{label}\PY{o}{=}\PY{l+s+s1}{\PYZsq{}}\PY{l+s+s1}{Cumulative explained variance}\PY{l+s+s1}{\PYZsq{}}\PY{p}{,} \PY{n}{color}\PY{o}{=}\PY{l+s+s1}{\PYZsq{}}\PY{l+s+s1}{r}\PY{l+s+s1}{\PYZsq{}}\PY{p}{)}
         \PY{n}{plt}\PY{o}{.}\PY{n}{axhline}\PY{p}{(}\PY{n}{y}\PY{o}{=}\PY{l+m+mi}{90}\PY{p}{,} \PY{n}{linewidth}\PY{o}{=}\PY{l+m+mi}{1}\PY{p}{,} \PY{n}{color}\PY{o}{=}\PY{l+s+s1}{\PYZsq{}}\PY{l+s+s1}{g}\PY{l+s+s1}{\PYZsq{}}\PY{p}{,} \PY{n}{linestyle}\PY{o}{=}\PY{l+s+s1}{\PYZsq{}}\PY{l+s+s1}{dashed}\PY{l+s+s1}{\PYZsq{}}\PY{p}{,} \PY{n}{label}\PY{o}{=}\PY{l+s+s1}{\PYZsq{}}\PY{l+s+s1}{90 }\PY{l+s+s1}{\PYZpc{}}\PY{l+s+s1}{\PYZsq{}}\PY{p}{)}
         \PY{n}{plt}\PY{o}{.}\PY{n}{axvline}\PY{p}{(}\PY{n}{x}\PY{o}{=}\PY{l+m+mi}{10}\PY{p}{,} \PY{n}{linewidth}\PY{o}{=}\PY{l+m+mi}{1}\PY{p}{,} \PY{n}{color}\PY{o}{=}\PY{l+s+s1}{\PYZsq{}}\PY{l+s+s1}{y}\PY{l+s+s1}{\PYZsq{}}\PY{p}{,} \PY{n}{linestyle}\PY{o}{=}\PY{l+s+s1}{\PYZsq{}}\PY{l+s+s1}{dashed}\PY{l+s+s1}{\PYZsq{}}\PY{p}{,} \PY{n}{label}\PY{o}{=}\PY{l+s+s1}{\PYZsq{}}\PY{l+s+s1}{10 principal components}\PY{l+s+s1}{\PYZsq{}}\PY{p}{)}
         \PY{n}{plt}\PY{o}{.}\PY{n}{title}\PY{p}{(}\PY{l+s+s1}{\PYZsq{}}\PY{l+s+s1}{Explained variance by different principal components}\PY{l+s+s1}{\PYZsq{}}\PY{p}{)}
         \PY{n}{plt}\PY{o}{.}\PY{n}{ylabel}\PY{p}{(}\PY{l+s+s1}{\PYZsq{}}\PY{l+s+s1}{Explained variance in percent}\PY{l+s+s1}{\PYZsq{}}\PY{p}{)}
         \PY{n}{plt}\PY{o}{.}\PY{n}{xlabel}\PY{p}{(}\PY{l+s+s1}{\PYZsq{}}\PY{l+s+s1}{Principal components}\PY{l+s+s1}{\PYZsq{}}\PY{p}{)}
         \PY{n}{plt}\PY{o}{.}\PY{n}{legend}\PY{p}{(}\PY{n}{loc}\PY{o}{=}\PY{l+s+s1}{\PYZsq{}}\PY{l+s+s1}{center right}\PY{l+s+s1}{\PYZsq{}}\PY{p}{)}
         \PY{n}{plt}\PY{o}{.}\PY{n}{show}\PY{p}{(}\PY{p}{)}
         
         \PY{n+nb}{print}\PY{p}{(}\PY{l+s+s1}{\PYZsq{}}\PY{l+s+s1}{Explained variance with 10 principal components: }\PY{l+s+si}{\PYZob{}\PYZcb{}}\PY{l+s+s1}{ }\PY{l+s+s1}{\PYZpc{}}\PY{l+s+s1}{\PYZsq{}}\PY{o}{.}\PY{n}{format}\PY{p}{(}\PY{n+nb}{sum}\PY{p}{(}\PY{n}{var\PYZus{}explained}\PY{p}{[}\PY{p}{:}\PY{l+m+mi}{10}\PY{p}{]}\PY{p}{)}\PY{p}{)}\PY{p}{)}
\end{Verbatim}

    \begin{center}
    \adjustimage{max size={0.9\linewidth}{0.9\paperheight}}{output_41_0.png}
    \end{center}
    { \hspace*{\fill} \\}
    
    \begin{Verbatim}[commandchars=\\\{\}]
Explained variance with 10 principal components: 89.13064299709966 \%

    \end{Verbatim}

    It is shown in the plot above, that 10 principal components explain
approximately 90\% of the total variance.

    \subsection{Data projection}\label{data-projection}

The 16 principal components represent a 16 dimensional feature space. We
can acheive 90 \% of the explained variance by projecting 10 principal
components onto a new feature space of 10 dimensions.

A projection matrix is constructed, which will represent a
10-dimensional feature space including the 10 first principal components
as columns.

    \begin{Verbatim}[commandchars=\\\{\}]
{\color{incolor}In [{\color{incolor}20}]:} \PY{n}{projection\PYZus{}mat} \PY{o}{=} \PY{n}{np}\PY{o}{.}\PY{n}{hstack}\PY{p}{(}\PY{p}{(}\PY{n}{eigen\PYZus{}pairs}\PY{p}{[}\PY{l+m+mi}{0}\PY{p}{]}\PY{p}{[}\PY{l+m+mi}{1}\PY{p}{]}\PY{o}{.}\PY{n}{reshape}\PY{p}{(}\PY{l+m+mi}{16}\PY{p}{,}\PY{l+m+mi}{1}\PY{p}{)}\PY{p}{,}
                                    \PY{n}{eigen\PYZus{}pairs}\PY{p}{[}\PY{l+m+mi}{1}\PY{p}{]}\PY{p}{[}\PY{l+m+mi}{1}\PY{p}{]}\PY{o}{.}\PY{n}{reshape}\PY{p}{(}\PY{l+m+mi}{16}\PY{p}{,}\PY{l+m+mi}{1}\PY{p}{)}\PY{p}{,}
                                    \PY{n}{eigen\PYZus{}pairs}\PY{p}{[}\PY{l+m+mi}{2}\PY{p}{]}\PY{p}{[}\PY{l+m+mi}{1}\PY{p}{]}\PY{o}{.}\PY{n}{reshape}\PY{p}{(}\PY{l+m+mi}{16}\PY{p}{,}\PY{l+m+mi}{1}\PY{p}{)}\PY{p}{,}
                                    \PY{n}{eigen\PYZus{}pairs}\PY{p}{[}\PY{l+m+mi}{3}\PY{p}{]}\PY{p}{[}\PY{l+m+mi}{1}\PY{p}{]}\PY{o}{.}\PY{n}{reshape}\PY{p}{(}\PY{l+m+mi}{16}\PY{p}{,}\PY{l+m+mi}{1}\PY{p}{)}\PY{p}{,}
                                    \PY{n}{eigen\PYZus{}pairs}\PY{p}{[}\PY{l+m+mi}{4}\PY{p}{]}\PY{p}{[}\PY{l+m+mi}{1}\PY{p}{]}\PY{o}{.}\PY{n}{reshape}\PY{p}{(}\PY{l+m+mi}{16}\PY{p}{,}\PY{l+m+mi}{1}\PY{p}{)}\PY{p}{,}
                                    \PY{n}{eigen\PYZus{}pairs}\PY{p}{[}\PY{l+m+mi}{5}\PY{p}{]}\PY{p}{[}\PY{l+m+mi}{1}\PY{p}{]}\PY{o}{.}\PY{n}{reshape}\PY{p}{(}\PY{l+m+mi}{16}\PY{p}{,}\PY{l+m+mi}{1}\PY{p}{)}\PY{p}{,}
                                    \PY{n}{eigen\PYZus{}pairs}\PY{p}{[}\PY{l+m+mi}{6}\PY{p}{]}\PY{p}{[}\PY{l+m+mi}{1}\PY{p}{]}\PY{o}{.}\PY{n}{reshape}\PY{p}{(}\PY{l+m+mi}{16}\PY{p}{,}\PY{l+m+mi}{1}\PY{p}{)}\PY{p}{,}
                                    \PY{n}{eigen\PYZus{}pairs}\PY{p}{[}\PY{l+m+mi}{7}\PY{p}{]}\PY{p}{[}\PY{l+m+mi}{1}\PY{p}{]}\PY{o}{.}\PY{n}{reshape}\PY{p}{(}\PY{l+m+mi}{16}\PY{p}{,}\PY{l+m+mi}{1}\PY{p}{)}\PY{p}{,}
                                    \PY{n}{eigen\PYZus{}pairs}\PY{p}{[}\PY{l+m+mi}{8}\PY{p}{]}\PY{p}{[}\PY{l+m+mi}{1}\PY{p}{]}\PY{o}{.}\PY{n}{reshape}\PY{p}{(}\PY{l+m+mi}{16}\PY{p}{,}\PY{l+m+mi}{1}\PY{p}{)}\PY{p}{,}
                                    \PY{n}{eigen\PYZus{}pairs}\PY{p}{[}\PY{l+m+mi}{9}\PY{p}{]}\PY{p}{[}\PY{l+m+mi}{1}\PY{p}{]}\PY{o}{.}\PY{n}{reshape}\PY{p}{(}\PY{l+m+mi}{16}\PY{p}{,}\PY{l+m+mi}{1}\PY{p}{)}\PY{p}{)}\PY{p}{)}
         \PY{n+nb}{print}\PY{p}{(}\PY{l+s+s1}{\PYZsq{}}\PY{l+s+s1}{Principal component space(10\PYZhy{}dimensional): }\PY{l+s+se}{\PYZbs{}n}\PY{l+s+s1}{\PYZsq{}}\PY{p}{,}\PY{n}{pd}\PY{o}{.}\PY{n}{DataFrame}\PY{p}{(}\PY{n}{projection\PYZus{}mat}\PY{p}{)}\PY{p}{)}
\end{Verbatim}

    \begin{Verbatim}[commandchars=\\\{\}]
Principal component space(10-dimensional): 
            0         1         2         3         4         5         6  \textbackslash{}
0  -0.362354 -0.155232 -0.327054 -0.061316  0.037905  0.081510 -0.053200   
1  -0.206547 -0.177457  0.272901  0.480791 -0.061195 -0.157182 -0.090268   
2  -0.158009 -0.094838  0.208406  0.126949  0.120085  0.160521  0.843224   
3  -0.252237  0.277675  0.006544 -0.138729 -0.284859 -0.135369  0.295252   
4  -0.222852  0.463064  0.164287  0.078790  0.287897  0.113896 -0.260482   
5  -0.317778 -0.135717  0.017333 -0.248506 -0.109607 -0.126009 -0.105331   
6  -0.385210 -0.240217  0.093722 -0.102664 -0.064023  0.037687 -0.064023   
7  -0.284547  0.505361  0.148158  0.023544  0.164534  0.054089 -0.127295   
8  -0.004622  0.161083 -0.088785  0.345420 -0.759305 -0.207488 -0.060707   
9  -0.340355 -0.257060  0.074307 -0.110103 -0.017139 -0.005135 -0.116929   
10 -0.057873 -0.037668 -0.067696 -0.066195  0.321987 -0.900841  0.076745   
11 -0.102572  0.114371 -0.667981  0.024169  0.089325  0.090690  0.057410   
12 -0.263037  0.365829  0.062501 -0.067118 -0.102478 -0.065726  0.172609   
13 -0.222104 -0.238787  0.272306  0.199927  0.049891  0.102415 -0.183407   
14 -0.092043 -0.007484 -0.290456  0.682436  0.255166 -0.006485  0.037680   
15 -0.317372 -0.122225 -0.300752 -0.063741 -0.081187  0.117026  0.013326   

           7         8         9  
0  -0.164751  0.071827 -0.029760  
1   0.065932  0.249215 -0.703490  
2  -0.252499 -0.273099 -0.034403  
3   0.313751  0.343233  0.076949  
4  -0.229883 -0.261221 -0.075917  
5   0.315229 -0.296164 -0.024031  
6   0.030401 -0.202532  0.163757  
7  -0.086317 -0.093371 -0.030542  
8  -0.310962 -0.320901  0.133912  
9   0.185893 -0.303659 -0.027260  
10 -0.230444 -0.009318  0.084763  
11 -0.154129  0.011778 -0.183168  
12  0.198561  0.309857  0.107281  
13 -0.342140  0.416090  0.517461  
14  0.464810 -0.110311  0.330770  
15 -0.255101  0.240690 -0.122173  

    \end{Verbatim}


    % Add a bibliography block to the postdoc
    
    
    
    \end{document}
